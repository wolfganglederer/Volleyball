\documentclass[]{book}
\usepackage{lmodern}
\usepackage{amssymb,amsmath}
\usepackage{ifxetex,ifluatex}
\usepackage{fixltx2e} % provides \textsubscript
\ifnum 0\ifxetex 1\fi\ifluatex 1\fi=0 % if pdftex
  \usepackage[T1]{fontenc}
  \usepackage[utf8]{inputenc}
\else % if luatex or xelatex
  \ifxetex
    \usepackage{mathspec}
  \else
    \usepackage{fontspec}
  \fi
  \defaultfontfeatures{Ligatures=TeX,Scale=MatchLowercase}
\fi
% use upquote if available, for straight quotes in verbatim environments
\IfFileExists{upquote.sty}{\usepackage{upquote}}{}
% use microtype if available
\IfFileExists{microtype.sty}{%
\usepackage{microtype}
\UseMicrotypeSet[protrusion]{basicmath} % disable protrusion for tt fonts
}{}
\usepackage[margin=1in]{geometry}
\usepackage{hyperref}
\hypersetup{unicode=true,
            pdftitle={Volleyball Übungen},
            pdfauthor={Wolfgang Lederer},
            pdfborder={0 0 0},
            breaklinks=true}
\urlstyle{same}  % don't use monospace font for urls
\usepackage{natbib}
\bibliographystyle{apalike}
\usepackage{longtable,booktabs}
\usepackage{graphicx,grffile}
\makeatletter
\def\maxwidth{\ifdim\Gin@nat@width>\linewidth\linewidth\else\Gin@nat@width\fi}
\def\maxheight{\ifdim\Gin@nat@height>\textheight\textheight\else\Gin@nat@height\fi}
\makeatother
% Scale images if necessary, so that they will not overflow the page
% margins by default, and it is still possible to overwrite the defaults
% using explicit options in \includegraphics[width, height, ...]{}
\setkeys{Gin}{width=\maxwidth,height=\maxheight,keepaspectratio}
\IfFileExists{parskip.sty}{%
\usepackage{parskip}
}{% else
\setlength{\parindent}{0pt}
\setlength{\parskip}{6pt plus 2pt minus 1pt}
}
\setlength{\emergencystretch}{3em}  % prevent overfull lines
\providecommand{\tightlist}{%
  \setlength{\itemsep}{0pt}\setlength{\parskip}{0pt}}
\setcounter{secnumdepth}{5}
% Redefines (sub)paragraphs to behave more like sections
\ifx\paragraph\undefined\else
\let\oldparagraph\paragraph
\renewcommand{\paragraph}[1]{\oldparagraph{#1}\mbox{}}
\fi
\ifx\subparagraph\undefined\else
\let\oldsubparagraph\subparagraph
\renewcommand{\subparagraph}[1]{\oldsubparagraph{#1}\mbox{}}
\fi

%%% Use protect on footnotes to avoid problems with footnotes in titles
\let\rmarkdownfootnote\footnote%
\def\footnote{\protect\rmarkdownfootnote}

%%% Change title format to be more compact
\usepackage{titling}

% Create subtitle command for use in maketitle
\newcommand{\subtitle}[1]{
  \posttitle{
    \begin{center}\large#1\end{center}
    }
}

\setlength{\droptitle}{-2em}
  \title{Volleyball Übungen}
  \pretitle{\vspace{\droptitle}\centering\huge}
  \posttitle{\par}
  \author{Wolfgang Lederer}
  \preauthor{\centering\large\emph}
  \postauthor{\par}
  \predate{\centering\large\emph}
  \postdate{\par}
  \date{2017-01-03}

\usepackage{booktabs}

\begin{document}
\maketitle

{
\setcounter{tocdepth}{1}
\tableofcontents
}
\chapter{Einleitung}\label{einleitung}

Gesammelte Volleyballübungen von Volleyfreuden e.V.

\chapter{Warm Up}\label{warm-up}

\section{Spielen mit Nachlaufen}\label{spielen-mit-nachlaufen}

Drei bis vier Spieler stellen sich in 2 Gruppen gegenüber auf. Der Ball
wird hin und her gespielt und man wechselt nach jedem Ballkontakt die
Seite (dem eigenen Ball nachlaufen) und stellt sich hinten an.

\textbf{Varianten}

\begin{itemize}
\tightlist
\item
  Oberes und unteres Zuspiel abwechseln
\item
  Oberes Zuspiel im Sprung
\item
  Jedesmal zum Ende des Laufens einen Hechtbagger oder Stemmschritt mit
  Sprung
\item
  Einmal selbst zuspielen dabei um 180° drehen und dann über Kopf zurück
\end{itemize}

\section{Spielen mit Nachlaufen
Diagonal}\label{spielen-mit-nachlaufen-diagonal}

Drei bis vie Spieler spielen sich den Ball mit Nachlaufen zu. Dabei
stehen sie diagonal im Feld und kreuzen den Weg mit einer zweiten
Gruppe. Zusätzlich zu den Varianten in \ref{spielen-mit-nachlaufen},
kann noch über Eck gelaufen werden.

\section{Paralell Spielen und Position
wechseln}\label{paralell-spielen-und-position-wechseln}

Je zwei Spieler stehen parallel und spielen den Ball gleichzeitig hin
und her. Nach jeder Ballberührung tauschen die Beiden (nicht paralleln
Spieler) die Position.

\textbf{Varianten}

\begin{itemize}
\tightlist
\item
  oberes und unters Zuspiel abwechseln
\item
  den Ball über Kreuz spielen
\item
  den Ball im Sprung zuspielen
\end{itemize}

\section{Spielen im Dreieck}\label{spielen-im-dreieck}

Die Spieler stellen sich im Dreieck auf. Der Ball wird in eine Richtung
gespielt und der Spieler läuft in die andere Richtung, usw.

\section{Ausbaggern}\label{ausbaggern}

Die Spieler stellen sich an beiden Enden des Feldes an. Der Ball wird
immer abwechselnd im unteren Zuspiel direkt über das Netz gespielt. Nach
jeder Ballberührung Lüft der Speiler auf die Andere Feldhälfte und
stellt sich hinten an. Es ist darauf zu achten, dass die andern Spieler
nicht behindert werden. Wenn ein Fehler (Netz, Aus, nicht im unteren
Zuspiel gespielt oder der Ball fällt auf den Boden) gemacht wird,
beginnt der Spieler der den Fehler beganngen hat von vorne und verliert
ein Leben. Hat ein Spieler alle seine Leben verbraucht, scheidet er aus
und macht sich solange individuell warm

\textbf{Varianten}

\begin{itemize}
\tightlist
\item
  Der Ball wird über einen Bodenkontakt eingeschlagen statt eingeworfen
\item
  Wenn der Ball auf den Boden fällt verliert der Spieler 2 Leben
\item
  Es wird nur auf den 3-Meter-Raum gespielt
\end{itemize}

\chapter{Einspielen}\label{einspielen}

\section{Technik}\label{technik}

\subsection{Angriff mit Nachwerfen}\label{angriff-mit-nachwerfen}

Um eine hohe Kadenz an Abwehr/Annahmebewegungen zu erreichen, hält der
Angriffsspieler 2 Bälle. Der erste Ball wird zuerst locker, im Laufe der
Übung immer fester angegriffen. Der zweite Ball wird dann relativ bald
hoch nachgeworfen (Dankeball). Der Verteidigungsspieler wehrt den
angegriffenen Ball ab und nimmt den geworfenen Ball danach an. Der
Angriffsspieler fängt den Abgewehrten Ball und greift den Dankeball
direkt an und wirft darauf den gefangen Ball nach \citep{hauser2016}.

\textbf{Varianten}

\begin{itemize}
\tightlist
\item
  Der Dankeball wird zu kurz, nach links oder nach rechts geworfen um
  eine Bewegung des Abwehrspielers zu erzwingen.
\item
  Die Angriffe werden härter oder gelobt.
\end{itemize}

\subsection{Ball ins Gesicht}\label{ball-ins-gesicht}

2 Spieler, 1 Ball. Der Ball wird vom Angriffsspieler auf den
Abwehrspieler geworfen. Der Angriffsspieler zielt auf den Bereich der
zwischen unterer und oberer Abwehr liegt, um die Entscheidung für eine
von beiden Techniken und der Ausführung zu trainieren.

\subsection{Werfen mit 3 Bällen}\label{werfen-mit-3-ballen}

Als Paar aufstellen, der Angriffsspieler hat 3 Bälle und wirft diese
nacheinander dem Abwehrspieler zu, der sie hoach abwehrt, sodass der
Angriffspieler die Bälle einen nach dem Anderen fangen und wieder
zurrück werfen kann.

\textbf{Varianten}

\begin{itemize}
\item
  Die Übung kann beliebig erschwert werden, indem die Bälle kürzer,
  länger oder seitlich versetzt geworfen werden. Es ist dann auf die
  Linie großer Zeh, Knie, Schulter zu achten
\item
  Es können aufeinander gestellte Bänke zwischen den Spielern
  aufgestellt werden um das Netzt zu simulieren
\end{itemize}

\subsection{Kurz, lang, kurz}\label{kurz-lang-kurz}

Paarweise. Spieler 1 spielt langen Ball zu Spieler 2 und läuft nach.
Spieler 2 spielt kurz vor sich zu Spieler 1, der wieder kurz zurück und
kehrt auf seine ursprüngliche Position zurück. Spieler 2 spielet nun
lang und läuft nun dem Ball nach und spielt den von Spieler 1 kurz
gespielten Ball zurück und kehrt dann wieder zum Ursprung zurück. Und so
weiter

\textbf{Varianten}

\begin{itemize}
\tightlist
\item
  mit Baggern
\item
  oberes Zuspiel im Sprung
\end{itemize}

\subsection{Lang, Mitte, Kurz}\label{lang-mitte-kurz}

Paarweise, ein Angriffsspieler mit Rücken zum Netz und ein
Abwehrspieler:

\begin{enumerate}
\def\labelenumi{\arabic{enumi}.}
\tightlist
\item
  Langer Angriff
\item
  Mittel gelegt
\item
  kurz gelegt
\item
  Ein Zwischenspiel und von vorne
\end{enumerate}

\subsection{Lang, kurz}\label{lang-kurz}

Angriffsspieler mit rücken zum Netz greift lang an auf Abwehrspieler.
Hohe Abwehr. Angriffsspieler legt den Ball kurz. Nach Zwischenspiel
wieder langer Angriff usw.

\textbf{Varianten}

\begin{itemize}
\tightlist
\item
  kurzer Ball muss im oberen Zuspiel gespielt werden
\item
  \ldots{}
\end{itemize}

\section{Koordination}\label{koordination}

\subsection{Parallel spielen mit 2
Bällen}\label{parallel-spielen-mit-2-ballen}

Zwei Spieler spielen parallel mit 2 Bällen. Die Technik sollte
vorgegeben werden. Zuerst oberes Zuspiel, unteres Zuspiel, oder
abwechselnd.

\subsection{Spiel mit 3 Bällen}\label{spiel-mit-3-ballen}

Zu zweit. Jeder spieler hält einen Ball in der Hand. Mit einem dritten
Ball wird hin und her gespielt. Der eigene Ball muss immer passend
hochgeworfen und wieder gefangen werden.

\subsection{Ein Spieler 2 Bälle}\label{ein-spieler-2-balle}

Ein Spieler hält einen Ball in jeder Hand das gegenüber spielt einen
Ball zu. Der Spieler mit den beiden Bällen in der hand wirft diese
rechtzeitig hoch um den vom Partner geworfenen Ball zu spielen und fängt
anschließend seine eigenen Bälle wieder.

Die Übung kann dadurch erschwert werden, dass kleinere Gegenstände
benutzt werden, z.B. Tennisbälle.

\subsection{Drei Spieler 4 Bälle}\label{drei-spieler-4-balle}

Wie Spielen mit Nachlaufen (\ref{spielen-mit-nachlaufen}), allerdings
hat jeder Spieler einen Ball in der Hand und wirft sich den Ball selbst
zu, spielt unter dessen den Spielball, fängt seinen eigenen Ball und
läuft dem Spielball nach.

\chapter{Angriff}\label{angriff}

\chapter{Block}\label{block}

\section{Block 1}\label{block-1}

\section{Block 2}\label{block-2}

\chapter{Annahme}\label{annahme}

\section{Schöne Annahmen}\label{schone-annahmen}

Mindesten 3er Gruppe. Ein Annahmespieler, ein Zuspieler/Fänger der Rest
schlägt auf. Es erfolgen Aufschläge auf den Annahmespieler, der zum
Fänger annimmt. Dieser zählt die schönen Annahmen mit. Nach einer fixen
Anzahl von schönen Annahmen wird gewechselt. Dies kann auf mehreren Feld
Hälften/Dritteln parallel ausgeführt werden.

\textbf{Varianten}

\begin{itemize}
\tightlist
\item
  Es wird nach jedem Ball die Position gewechselt (dem Ball nachlaufen).
\item
  Die ersten x Aufschläge sind leicht und müssen im oberen Zuspiel
  angenommen werden.
\item
  Falls mehrere Aufschläger pro Gruppe vorhanden sind, können die
  letzten Aufschläge mit mehr Risiko gemacht werden. (Wartezeit
  beachten).
\item
  Bei 2 Gruppen können Aufschläge auch diagonal erfolgen.
\item
  Man kann auch diagonal dem Ball nachlaufen.
\item
  Aufschläge werden von einem niedrigen Kasten gemacht um härtere
  Aufschläge zu simulieren.
\end{itemize}

\chapter{Komplex}\label{komplex}

\section{Diagonalspiel}\label{diagonalspiel}

\section{Kreisel}\label{kreisel}

\section{Diagonal Angriff mit
nachlaufen}\label{diagonal-angriff-mit-nachlaufen}

\section{Longline Pritschen, diagonal Baggern mit
nachlaufen}\label{longline-pritschen-diagonal-baggern-mit-nachlaufen}

\chapter{Positionen}\label{positionen}

\section{Aufstellung}\label{aufstellung}

\bibliography{packages,book}


\end{document}
